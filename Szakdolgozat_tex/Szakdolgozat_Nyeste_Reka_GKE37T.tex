\documentclass{thesis-ekf}
\usepackage[T1]{fontenc}
\PassOptionsToPackage{defaults=hu-min}{magyar.ldf}
\usepackage[magyar]{babel}
\begin{document}
	\institute{Matematikai és Informatikai Intézet}
	\title{Programozási nyelveket tanító alkalmazás}
	\author{Nyeste Réka\\Programtervező informatikus BSc.}
	\supervisor{Dr. Tajti Tibor Gábor\\egyetemi docens}
	\city{Eger}
	\date{2023}
	\maketitle
	\tableofcontents
	\chapter{Bevezetés}
	\section{A szakdolgozatom címe és technikai háttere röviden}
	Azért választottam ezt a témát, mert szerintem fontos az, hogy a kezdő programozók megfelelő alapokat kapjanak a fejlődésük érdekében, illetve akit egy kicsit is érdekel a programozás ne riadjon vissza, hanem bátran kezdje el tanulni.
	
	Az applikációmat Java nyelven írtam, ami manapság az egyik legelterjedtebb Objektum Orientált programozási nyelv, ezért szerettem volna elmélyíteni tudásomat ezen a nyelven. Programozási környezetnek az Android Studiot használtam, amiben rendkívül egyszerű programozni. Az Jetbrains IntelliJ IDEA-n alapszik ez a környezet és kifejezetten Android-ra tervezett programok fejlesztésére lett kialakítva. Segítségemre volt az egyik alap funkciója, ami egy virtuális eszköz beállítását teszi lehetővé, ilyenek lehetnek például: tévé, tablet, telefon, okos óra vagy akár egy autó beépített kezelőfelülete is. Fontos még kiemelnem, hogy törekedtem olyan programot írni, mely sok készüléken elérhető, emiatt választottam az Android 5.0 Lollipop verziót, ami szinte minden eszközön elérhető.
	
	Programom nyomon követését a GitHub és GitKraken verziókövető rendszerekkel valósítottam meg. Azért ezeket választottam, mert jól ismerem a működésüket, könnyű bennük visszakeresni az esetlegesen felmerülő hibáimat és gyorsan is lehet azokat kijavítani, valamint összehangoltan lehet dolgozni az Android Studioval is, annak beépített funkcióinak a segítségével. 
	
	\section{Képernyő tervek}
	\section{Adatbázis tervek}
	\section{Kidolgozandó terveim}
	\begin{thebibliography}{1}
		\bibitem{cimke} \textsc{Szerző}: Cím, Kiadó, Hely, évszám.
	\end{thebibliography}
\end{document}