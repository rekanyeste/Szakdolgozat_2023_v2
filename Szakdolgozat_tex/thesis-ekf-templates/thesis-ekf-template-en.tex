\documentclass{thesis-ekf}
\usepackage[T1]{fontenc}
\usepackage[english]{babel}

\usepackage{amsthm}
\newtheorem{theorem}{Theorem}[chapter]
\theoremstyle{definition}
\newtheorem{definition}[theorem]{Definition}
\theoremstyle{remark}
\newtheorem{remark}[theorem]{Remark}

\begin{document}
\institute{Institute of Mathematics and Informatics}
\title{Title}
\author{Author's name\\ major}
\supervisor{Supervisor's name\\ post}
\city{Eger}
\date{2022}
\maketitle
\tableofcontents

\chapter*{Introduction}
\addcontentsline{toc}{chapter}{Introduction}
Let us suppose that the noumena have nothing to do with necessity, since knowledge of the Categories is a posteriori. Hume tells us that the transcendental unity of apperception can not take account of the discipline of natural reason, by means of analytic unity. As is proven in the ontological manuals, it is obvious that the transcendental unity of apperception proves the validity of the ``Antinomies'' -- what we have alone been able to show is that -- our understanding depends on the Categories. It remains a mystery why the Ideal stands in need of reason. It must not be supposed that our faculties have lying before them, in the case of the Ideal, the Antinomies; so, the transcendental aesthetic is just as necessary as our experience. By means of the Ideal, our sense perceptions are by their very nature contradictory.
\cite{Knuth}

\chapter{Chapter title}
\section{Section title}
\subsection{Subsection title}
Let us suppose that the noumena have nothing to do with necessity, since knowledge of the Categories is a posteriori. Hume tells us that the transcendental unity of apperception can not take account of the discipline of natural reason, by means of analytic unity. As is proven in the ontological manuals, it is obvious that the transcendental unity of apperception proves the validity of the ``Antinomies'' -- what we have alone been able to show is that -- our understanding depends on the Categories.
\cite[p.~102]{Knuth}

It remains a mystery why the Ideal stands in need of reason. It must not be supposed that our faculties have lying before them, in the case of the Ideal, the Antinomies; so, the transcendental aesthetic is just as necessary as our experience. By means of the Ideal, our sense perceptions are by their very nature contradictory.
\cite{Knuth,Manning}

\begin{theorem}
Text.
\end{theorem}

\begin{proof}
Text.
\end{proof}

\begin{definition}
Text.
\end{definition}

\begin{remark}
Text.
\end{remark}

\chapter*{Summary}
\addcontentsline{toc}{chapter}{Summary}
Let us suppose that the noumena have nothing to do with necessity, since knowledge of the Categories is a posteriori. Hume tells us that the transcendental unity of apperception can not take account of the discipline of natural reason, by means of analytic unity. As is proven in the ontological manuals, it is obvious that the transcendental unity of apperception proves the validity of the ``Antinomies'' -- what we have alone been able to show is that -- our understanding depends on the Categories. It remains a mystery why the Ideal stands in need of reason. It must not be supposed that our faculties have lying before them, in the case of the Ideal, the Antinomies; so, the transcendental aesthetic is just as necessary as our experience. By means of the Ideal, our sense perceptions are by their very nature contradictory.

\begin{thebibliography}{2}
\addcontentsline{toc}{chapter}{\bibname}
\bibitem{Knuth}
\textsc{Donald Ervin Knuth}: \emph{Deformation modelling tracking animation and applications}, Berlin, Heidelberg, Springer, 2001.
\bibitem{Manning}
\textsc{Christopher Manning, Prabhakar Raghavan, Hinrich Sch\"{u}tze}: \emph{Introduction to Information Retrieval}, New York, USA, Cambridge University Press, 2008.
\end{thebibliography}
\end{document}